\documentclass{article}
\usepackage{amsmath}

\title{Othello Heuristics for Evaluation Functions}
\author{Vlad-Mihai Bogdan}
\date{\today}

\begin{document}

\maketitle

\section*{Introduction}
\subsection*{Thing to consider when designing an evaluation function}

\quad Othello is a game with a $8 x 8$ board, so the number of possible games can pe weakly estimated to be $3^{64}$. 
This means that the search space is huge, and the evaluation function is crucial for the performance of the engine.

Creating a good heuristic is a difficult task, as it requires a deep understanding of the game. 
And since I don't have, I will just try to focus on the most important aspects of the game.

The factors that I will consider are:
\begin{itemize}
  \item Mobility
  \item Stability
  \item Coin Parity
  \item Corners and the weight of each square
\end{itemize}

\subsection*{Mobility}
\quad \textbf{Mobility} is the number of legal moves that a player has. 
As a general rule, the less moves a player has, the worse the position is, since it means that the player may be forced to make a bad move.

Mobility comes in two forms: \textbf{Actual mobility} and \textbf{Potential mobility}. 
Actual mobility is simply calculated by counting the number of legal moves that a player has.
Potential mobility is the number of squares that are adjacent to at least one of opponent's discs.

\subsection*{Stability}
\quad \textbf{Stability} is the number of discs that can't be flipped. 
Such discs are called \textbf{stable discs}. For example, the corners are stable discs, since they can't be flipped.
Since the goal of the game is to have as many discs as possible, knowing the number of discs that will defenetly be yours is important.

\subsection*{Coin Parity}
\quad \textbf{Coin Parity} is the difference between the number of discs of the player and the number of discs of the opponent.
This is defenetly as important as the other factors, since a single move can flip up to $18$ discs.

\subsection*{Corners and the weight of each square}
\quad The \textbf{corners} are the most important squares in the game, since they are stable. 
This gives us the ideea that the neighbouring squares of the corners are not good to capture, as they would lead to the loss of the corner.
Using this ideea, we will assign a weight to each square, based on its importance.

\section*{Static weight evaluation function}
\quad As we said, different squares have different importance. 
For a simple and fast evaluation function, we will assign a weight to each square, based on its importance.
The result of the function will be something like this:

\[
  \text{eval}(b) = 
  \begin{cases}
    -\infty & \text{if MIN won} \\
    \infty & \text{if MAX won} \\
    \sum_{(i,j) \in b_{MAX}} \text{weight}_{i,j} - \sum_{(i,j) \in b_{MIN}} \text{weight}_{i,j} & \text{otherwise}
  \end{cases}
\]
where $b_{MAX}$ is the set of cells occupied by MAX in $b$. Same for $b_{MIN}$.

The weights of the squares I will be using are:
\[
  \begin{bmatrix}
    99 & -8 & 8 & 6 & 6 & 8 & -8 & 99 \\
    -8 & -24 & -4 & -3 & -3 & -4 & -25 & -8 \\
    8 & -4 & 7 & 4 & 4 & 7 & -4 & 8 \\
    6 & -3 & 4 & 0 & 0 & 4 & -3 & 6 \\
    6 & -3 & 4 & 0 & 0 & 4 & -3 & 6 \\
    8 & -4 & 7 & 4 & 4 & 7 & -4 & 8 \\
    -8 & -24 & -4 & -3 & -3 & -4 & -25 & -8 \\
    99 & -8 & 8 & 6 & 6 & 8 & -8 & 99 \\
  \end{bmatrix}
\]

\section*{?}

\end{document}
